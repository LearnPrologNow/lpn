
\chapter*{Prolog Environments}\label{Appendix:prologs}


Several Prolog environments are available, and probably the best idea
is simply to google what's available. But we list here four of the
more widely used systems.

\begin{itemize}

\item \textbf{SWI-Prolog}\\
      A Free Software Prolog environment, licensed under the Lesser GNU public
      license.  This popular interpreter was developed by Jan
      Wielemaker. \\ \url{http://www.swi-prolog.org/}

\item \textbf{SICStus Prolog}\\
      Industrial strength Prolog environment from the Swedish Institute of
      Computer Science.\\ \url{http://www.sics.se/sicstus/}

\item \textbf{YAP Prolog}\\
      A Prolog compiler developed at the Universidade do Porto and Universidade
      Federa do Rio de Janeiro.  Free for use in academic environments. \\
      \url{http://www.ncc.up.pt/~vsc/Yap/}

\item \textbf{Ciao Prolog}\\
      Another Prolog environment available under the GNU public license,
      developed at the Universidad Polit\'{e}cnica de Madrid.\\
      \url{http://clip.dia.fi.upm.es/Software/Ciao/}

\end{itemize}
