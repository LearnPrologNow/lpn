
\chapter*{Further Reading}\label{Appendix:books}

While we think \textit{Learn Prolog Now!} is a good first book on Prolog, it
certainly shouldn't be the last one you look at. To help you take the next
step, we have listed, with comments, some of our favourite Prolog textbooks,
and Prolog-based books on Artificial Intelligence (AI) and Natural Language
Processing (NLP).


\subsection*{Prolog textbooks}

\begin{itemize}

\item Bratko (1990): \textit{Prolog Programming for Artificial
Intelligence}. Addison-Wesley.  We strongly recommend this book. If you liked
\textit{Learn Prolog Now!}  we think you'll find this a natural followup. Its
strong point is the wide variety of programming styles and applications it
considers.  This is a big book, and it will take you quite a while to work
through it. But if you do so, you'll soon be writing very substantial Prolog
programs indeed, and you'll learn a lot about AI along the way.

\item Clocksin (2003): \textit{Clause and Effect: Prolog Programming
for the Working Programmer}. Springer.  Strongly recommended. If you
want a concise practically oriented follow up to \textit{Learn Prolog
Now!}  that will really hone your Prolog skills, you can't do better
than this. It explains some interesting theory, but its real strength
is that it is based around a collection of worksheets. Solve the
problems they contain, and you'll soon be flying.

\item Clocksin and Mellish (1987): \textit{Programming in Prolog}. Springer.
  This was one of the earliest, if not the earliest, textbook on Prolog
  programming. It won't take you far beyond \textit{Learn Prolog Now!}, but it
  is clearly written, and its discussions of DCGs, and of the link between
  logic and Prolog, are accessible and worth looking at.


\item O'Keefe (1990): \textit{Craft of Prolog}. MIT Press.  This is the book
you should read when you're convinced that you know all about Prolog and have
nothing left to learn. Unless you truly are a Prolog guru, you will swiftly
learn that there are far deeper levels of Prolog expertise than you suspected,
and that you still have a great deal to master. Superb.

\item Sterling (1994): \textit{The Art of Prolog}. MIT Press.  In \textit{Learn
Prolog Now!} we don't say much about the abstract idea of logic programming. If
the little we have said has wakened your interest, this is the
book to go for next. Clearly written, it will give you a good grounding in the
basic theory of logic programming, and link it to the practical world of
Prolog.

\end{itemize}






\subsection*{Applying Prolog in AI and NLP}

\begin{itemize}

\item Blackburn and Bos (2005): \textit{Representation and Inference
      for Natural Language. A First Course in Computational
      Semantics}. CSLI Lecture Notes. Introduces natural language
      semantics from a computational perspective using Prolog as the
      implementation language.  \textit{Learn Prolog Now!} was
      originally intended to be an appendix to this book.


\item Covington (1994): \textit{Natural Language Processing for Prolog
Programmers}. Prentice-Hall. Solid, well-written book on NLP that uses Prolog
as the implementation language. If you haven't done any NLP before, and want to
put your Prolog to work, this is a good place to start.

\item Pereira and Shieber (1987): \textit{Prolog and Natural Language
  Analysis}. CSLI Lecture Notes. A classic. Several generations of PhD
  students have cut their teeth on this one. Required reading.

\item Reiter (2001): \textit{Knowledge in Action: Logical Foundations for
 Specifying and Implementing Dynamical Systems}. MIT Press.  This book
 examines, extends, and implements the Situation Calculus, a well known AI
 formalism for representing and reasoning about changing information.  It's an
 important book, and may not be completely accessible if you don't have some
 theoretical background. But as an example of how Prolog can be put to work, it
 takes some beating.

\item Shoham (1994): \textit{Artificial Intelligence Techniques in
Prolog}. Morgan Kaufman.  Discusses and implements a wide range of AI
problem-solving techniques and concepts, including depth-first search,
breadth-first search, best-first search, alpha-beta minimax, forward chaining,
production systems, reasoning with uncertainty, and STRIPS.

\end{itemize}
